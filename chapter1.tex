\chapter{Introduction}
\label{introchap}

Lunar swirls are one of the most enigmatic features on the lunar regolith.  They are identified by their sinuous albedo patterns and high reflectivity of visible and near infrared radiation \cite{denevi2016distribution}.  The Reiner Gamma swirl (RG) is the largest known swirl of its type and has a unique tadpole formation spanning 100 $km$ across Oceanus Procellarum \cite{denevi2016distribution}.  Like all lunar swirls, RG is accompanied by a relatively strong lunar magnetic anomaly (LMA) \cite{blewett2011lunar}. \par  

The origin of LMAs is a controversial topic.  

The origin and geometries of these LMA fields are not well understood, but their surface strength is believed to be on the order of thousands of nanoTeslas ($nT$) \cite{bibid}, while orbital data taken 30$km$ above the surface has measured fields on the order of 10$nT$ \cite{blewett2011lunar}.  Understanding swirl morphology can inform us of the unique surface characteristics of many airless bodies.  Presently, there are two main theories concerning the formation of lunar swirls, the solar wind standoff model and the dust lofting model, neither of which sufficiently explains all of the observed phenomenon associated with lunar swirls \cite{garrick2011spectral}\cite{blewett2011lunar}.  \par

In order to consider the efficacy of these competing theories, it is first important to understand the space weathering process that darkens the lunar surface material.  The two leading theories concerning the process of space weathering are both dependent on the formation of submicroscopic metallic iron (SMFe).  First are the effects of solar wind sputtering, in which the solar wind will saturate the lunar surface with hydrogen ions, leading to a physical reaction with the lunar surface to form SMFe \cite{hapke1973darkening}.  The second process, known as the vapor deposition model, suggests that meteorite bombardment will vaporize glass agglutinates, thereby depositing iron bearing silicate vapors on surrounding dust particles \cite{hapke2001space}.  It is generally considered, however, that the most likely explanation is a combination of both effects.  First, the implantation of hydrogen due to solar wind will produce an oxide coating in the lunar regolith, then this coating will produce an iron-bearing silicate vapor upon a collision event.  This vapor will then be deposited on the surrounding dust as SMFe \cite{hapke2001space}. \par

The solar wind standoff model therefore attempts to explain how a LMA might limit the formation of SMFe.  The theory is dependent on two facotrs: (1) the influx or uncovering of optically immature subsurface dust in swirl regions, (2) the presence of a magnetic field horizontal to the lunar surface strong enough to reflect the solar wind, thereby preventing solar wind sputtering \cite{glotch2015formation}.  There are two main proposals as to how a swirl might have an influx or uncovering of subsurface dust.  First is due to the observed proximity correlation between lunar swirls and fresh impact craters \cite{}.  This suggests that these collisions may have brought fresh dust into the swirl region.  Second, it has been proposed that micrometeorite bombardment might superficially scour the lunar surface, uncovering fresh dust while also contributing to the formation of the surrounding magnetic field \cite{}.  The crater proposal is limited by the fact that not all swirls, asd and asdf for example, have a fresh impact crater nearby \cite{}.  The micrometeorite bombardment hypothesis also has a limitation in that micrometeorite bombardment is known to produce many of the lunar regolith's agglutinates \cite{}.  This, however, is inconsistent with ultraviolet reflectance observations of the Lunar Reconnaissance Orbiter Camera (LROC) Wide Angle Camera (WAC), which found an absence of surface agglutinates in many swirl regions \cite{denevi2014characterization}.  The most significant shortcoming of the solar wind standoff model, however, is a lack of demonstrable reflection of the solar wind by any LMA other than the one found near Reiner Gamma \cite{lucey2006understanding}. 

This suggests that any space weathering effects that might be observed in the region of lunar swirls would be due to the vapor deposition that occurs as a result of micrometeorite bombardment \cite{glotch2015formation}.  This is consistent with ultraviolet reflectance observations of the Reiner Gamma swirl from the Lunar Reconnaissance Orbiter Camera (LROC) Wide Angle Camera (WAC), which found optically immature reflectance patterns around multiple lunar swirls as well as an absence of surface agglutinates \cite{denevi2014characterization}.  This suggests that micrometeorite bombardment, without the presence of solar radiation, might do more to excavate the lunar surface than to form glassy silicate melts \cite{denevi2014characterization}.  This excavation might also uncover optically immature dust just below the top layer in the lunar regolith, thereby contributing to the formation of lunar swirls \cite{}.  These observations do provide strong evidence for the solar wind standoff model, but the extent to which the solar wind is necessary in the soil darkening process is not well understood.  The reflection of the solar wind, however, has not been sufficiently demonstrated for LMAs weaker than the one found at Reiner Gamma \cite{lucey2006understanding}.  Therefore, the solar wind standoff model is not yet sufficient in explaining the process of swirl morphology.   \par

The other leading theory in swirl morphology is the dust lofting model.  





This invites continued exploration into all the possible contributions to their formation.  It is therefore being proposed that magnetized dust grains, lofted off the lunar surface due to photoelectric emissions \cite{abbas2007lunar}, might rotate due to the surrounding LMA, thereby producing a unique landing pattern that might affect the photometric properties of these lunar swirls.  This hypothesis is motivated by the photometric properties of swirls that indicate smoother dust structures than the lunar background \cite{kaydash2009photometric}, as well as the increase of 2 wt\% in the iron content of surface dust observed at the RG swirl compared to the surrounding area \cite{pinet2000local}.  In order to test this hypothesis, we have developed a simulation that models the rotation of "needle-like" particles as they travel through a simulated lunar dipole field that has been compressed by the solar wind \cite{deca2015general}.